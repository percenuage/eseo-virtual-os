
\documentclass[utf8, 14pt]{beamer}

\include{Configuration}

%----------------------------------------------------------------------------------------
%	TITLE PAGE
%----------------------------------------------------------------------------------------

\title[Design Patterns]{Design Patterns}
\subtitle{OS Virtuel}

\author[Axel G. | Audoin De C.]{Axel \textsc{Gendillard} \and Audoin \textsc{De Chantérac}}
\institute[]
{
	Étudiant ingénieur\\ 
	Groupe ESEO \\
	%\medskip
	%\textit{axel.gendillard@reseau.eseo.fr}
}

\date{\today} 

%\logo{\includegraphics[scale=0.3]{logo}}

\titlegraphic{%
%\includegraphics[scale=0.55]{logo} \hspace{1cm}~%
%\includegraphics[scale=0.6]{eseo}
}


\begin{document}

\begin{frame}
\titlepage
\end{frame}

\begin{frame}{Sommaire}
\tableofcontents
\end{frame} 

%----------------------------------------------------------------------------------------
%	PRESENTATION SLIDES
%----------------------------------------------------------------------------------------

\section{Incrément 1 : Singleton}
	
	\begin{frame}{Incrément 1 : Singleton}
		\begin{block}{Objectifs}
		\begin{itemize}
		\item Créer un nouvel utilisateur à chaque validation;
		\item Avoir un utilisateur unique par login;
		\item Créer plusieurs terminaux par utilisateur;
		\item Interdir la création d'utilisateur hors de la fenêtre.
		\end{itemize}
		\end{block}
		
		\begin{exampleblock}{Patron Singleton}
		\begin{itemize}
		\item Garantir l'unicité de l'instance d'une classe;
		\item Accéder globalement à l'instance.
		\end{itemize}
		\end{exampleblock}					
	\end{frame}

	\begin{frame}{Incrément 1 : Singleton - UML}
		\begin{figure}[!h]
		\centering
		\includegraphics[width=\textwidth]{../uml/uml-singleton}
		\end{figure}		
	\end{frame}


\section{Incrément 2 : Observer}
	
	\begin{frame}{Incrément 2 : Observer}
		\begin{block}{Objectifs}
		\begin{itemize}
		\item Synchroniser les historiques des terminaux;
		\item Mettre à jour tous les terminaux lors de l'entrée d'une commande.
		\end{itemize}
		\end{block}
		
		\begin{exampleblock}{Patron Observer}
		\begin{itemize}
		\item Permet de mettre à jour les objets observés en fonction de l'objet observateur.
		\end{itemize}
		\end{exampleblock}					
	\end{frame}

	\begin{frame}{Incrément 2 : Observer - UML}
		\begin{figure}[!h]
		\centering
		\includegraphics[width=\textwidth]{../uml/uml-observer}
		\end{figure}		
	\end{frame}



\section{Incrément 3 : Visiteur}
	
	\begin{frame}{Incrément 3 : Visiteur}
		\begin{block}{Objectifs}
		\begin{itemize}
		\item Interpréter des commandes du terminal (ls et cat) pour chaque type de noeud;
		\item Utiliser le patron Visitor.
		\end{itemize}
		\end{block}
		
		\begin{exampleblock}{Patron Visiteur}
		\begin{itemize}
		\item Sépare l'algorithme de la structure de données;
		\item Chaque élément implémente une méthode d'acceptation de visiteur;
		\item Chaque visiteur implémente une méthode \emph{visit()} pour chaque élément (file, folder et link).
		\end{itemize}
		\end{exampleblock}					
	\end{frame}

	\begin{frame}{Incrément 3 : Visiteur - UML}
		\begin{figure}[!h]
		\centering
		\includegraphics[width=\textwidth]{../uml/uml-visitor}
		\end{figure}		
	\end{frame}


\section{Incrément 4 : Commande}
	
	\begin{frame}{Incrément 4 : Commande}
		\begin{block}{Objectifs}
		\begin{itemize}
		\item Réutiliser les traitements interprétés par le terminal;
		\item Exprimer de manière unique le traitement des commandes ls et cat.
		\end{itemize}
		\end{block}
		
		\begin{exampleblock}{Patron Commande}
		\begin{itemize}
		\item Isole une requête;
		\item Ces requêtes peuvent provenir de plusieurs émetteurs (\emph{TerminalOS} et \emph{ExplorerOS});
		\item Ces émetteurs doivent produire la même requête;
		\item Ces requêtes doivent pouvoir être annulées.
		\end{itemize}
		\end{exampleblock}					
	\end{frame}

	\begin{frame}{Incrément 4 : Commande - UML}
		\begin{figure}[!h]
		\centering
		\includegraphics[width=0.78\textwidth]{../uml/uml-command}
		\end{figure}		
	\end{frame}

\section{Incrément 5 : Commande ++}
	
	\begin{frame}{Incrément 5 : Commande ++}
		\begin{block}{Objectifs}
		\begin{itemize}
		\item RM : Supprimer un dossier, un fichier ou un lien;
		\item MKDIR : Créer un dossier;
		\item TOUCH : Créer un fichier;
		\item LN : Créer un lien.
		\end{itemize}
		\end{block}
		
		\begin{exampleblock}{Patron Commande}
		\begin{itemize}
		\item Isole une requête;
		\item Ces requêtes peuvent provenir de plusieurs émetteurs (\emph{TerminalOS} et \emph{ExplorerOS}) ayant le même focntionnement;
		\item Ces requêtes doivent pouvoir être annulées.
		\end{itemize}
		\end{exampleblock}					
	\end{frame}

	\begin{frame}{Incrément 5 : Commande - UML}
		\begin{figure}[!h]
		\centering
		\includegraphics[width=0.78\textwidth]{../uml/uml-command}
		\end{figure}		
	\end{frame}

\section{Incrément 6 : Proxy}
	
	\begin{frame}{Incrément 6 : Proxy}
		\begin{block}{Objectifs}
		\begin{itemize}
		\item Lecture seule : Pas de modification des données des utilisateurs ls et cat sont autorisées;
		\item Lecture/écriture : Les commandes mkdir, touch, ln et rm sont autorisées.
		\end{itemize}
		\end{block}
		
		\begin{exampleblock}{Patron Proxy}
		\begin{itemize}
		\item Rôle d'aiguilleur en fonction des droits de l'utilisateur;
		\item Couche d'abstraction entre le client et la commande.
		\end{itemize}
		\end{exampleblock}					
	\end{frame}

	\begin{frame}{Incrément 6 : Proxy - UML}
		\begin{figure}[!h]
		\centering
		\includegraphics[width=\textwidth]{../uml/uml-proxy}
		\end{figure}		
	\end{frame}

	\section{Bonus : Gestion de l'historique au clavier}
	
	\begin{frame}{Bonus : Gestion de l'historique au clavier}
		\begin{block}{Objectifs}
		\begin{itemize}
		\item Implémenter un gestionnaire d’historique au clavier (flèche du haut et du bas)
		\item Utiliser la classe \emph{java.awt.event.KeyAdapter}
		\end{itemize}
		\end{block}
		
		\begin{exampleblock}{Implémentation}
		\begin{itemize}
		\item Création classe \emph{KeyBoardListener} qui hérite de \emph{KeyAdapter}
		\item Création d'un \emph{listener} sur les événement du clavier
		\item La méthode \emph{handleCommand()} est appelée via la touche \emph{Enter}.
		\end{itemize}
		\end{exampleblock}					
	\end{frame}

	\begin{frame}{Bonus : UML}
		\begin{figure}[!h]
		\centering
		\includegraphics[width=0.95\textwidth]{../uml/uml-keyboard}
		\end{figure}		
	\end{frame}

	\begin{frame}{Bonus : Un peu de code pour finir}
		\begin{figure}[!h]
		\centering
		\includegraphics[width=\textwidth]{../uml/code}
		\end{figure}		
	\end{frame}


\end{document} 
